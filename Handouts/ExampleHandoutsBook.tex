\documentclass{book}
\title{Example Book for lecture  handouts}

\usepackage{pdfpages}
\usepackage{url}
\makeindex

\begin{document}
\maketitle
\tableofcontents

\chapter*{Introduction}
\label{chap_intro}

This book is created with the ``LaTeX Handouts Builder'' Python script.

\noindent More info: \url{https://github.com/jeroendoggen/latex-handouts-builder}

\noindent The script is still a work in progress. The features are based on a similar Bash script that I have been using for several years.

\section*{Program flow}

\begin{itemize}
\item Run one command 'build.py'
\item Several sets of LaTeX-beamer slides are build:
\item pdf documents are created
\item Different versions: 4/6 slides per page, handout version with printer friendly colors
\item These slide handouts are merged in one 'handouts-LaTeX-book' with a title page, introduction, table of contents,...
\item All these documents are compressed in a single .zip archive. (and uploaded to the course website)
\end{itemize}


\part{The First Chapter}
\includepdf[pages={-}]{chap1-6pp.pdf}

\part{The Second Chapter}
\includepdf[pages={-}]{chap2-6pp.pdf}

\part{The Third Chapter}
\includepdf[pages={-}]{chap3-6pp.pdf}

\part{The Fourth Chapter}
\includepdf[pages={-}]{chap4-6pp.pdf}

\part{The Fifth Chapter}
\includepdf[pages={-}]{chap5-6pp.pdf}

\part{The Sixth Chapter}
\includepdf[pages={-}]{chap6-6pp.pdf}

\part{The Seventh Chapter}
\includepdf[pages={-}]{chap7-6pp.pdf}

\part{The Eighth Chapter}
\includepdf[pages={-}]{chap8-6pp.pdf}


\end{document}
 


