\documentclass{book}
% \title{}
% \author{Jeroen Doggen}

\title{\LARGE \textbf{\uppercase{ Example Book for Lecture  Handouts}}}

\author{\textbf{Jeroen Doggen}  \\
        \texttt{jeroendoggen@gmail.com}
}

\usepackage{pdfpages}
\usepackage{url}
\makeindex

\begin{document}
\maketitle
\tableofcontents

\chapter*{Introduction}
\label{chap_intro}

This book is created with ''\LaTeX -Handouts Builder'', a Python script to create \LaTeX-beamer based course handouts.

\noindent More info: \url{http://jeroendoggen.github.com/latex-handouts-builder/}


\section*{What is it does:}

\begin{itemize}
\item Build multiple LaTeX beamer slide sets with one command: ``python build.py``
\item Convert the slides to a printer-friendly format (no slide transitions, 6 slides per page, less colors)
\item Build a main "course handouts book" with all the slides (one chapter per beamer slide set, title page, introduction, table of contents,...)
\item Create a .zip archive with all the documents
\end{itemize}

\noindent The scripts starts multiple processes in parallel to build all slide sets at the same time. This results in a big speedup on multi-core systems.

\part{The First Chapter}
\includepdf[pages={-}]{chap1_handout.pdf}

\part{The Second Chapter}
\includepdf[pages={-}]{chap2_handout.pdf}

\part{The Third Chapter}
\includepdf[pages={-}]{chap3_handout.pdf}

\part{The Fourth Chapter}
\includepdf[pages={-}]{chap4_handout.pdf}

\part{The Fifth Chapter}
\includepdf[pages={-}]{chap5_handout.pdf}

\part{The Sixth Chapter}
\includepdf[pages={-}]{chap6_handout.pdf}

\part{The Seventh Chapter}
\includepdf[pages={-}]{chap7_handout.pdf}

\part{The Eighth Chapter}
\includepdf[pages={-}]{chap8_handout.pdf}

\end{document}



